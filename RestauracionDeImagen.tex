\documentclass{article}
\usepackage{hyperref} % Para enlaces en el índice

\begin{document}

\tableofcontents

\section{Introducción}
\subsection{Motivación}
\subsection{Objetivos}
\subsection{Organización del Documento}

\section{Fundamentos Teóricos}
\subsection{Procesamiento Digital de Imágenes}
\subsubsection{Reducción de ruido (mínimo, máximo, mediana, anisotrópico)}
\subsubsection{Corrección de color y tonos (dithering, mapeo de color)}
\subsubsection{Interpolación y superresolución}
\subsection{Inteligencia Artificial en Restauración de Imágenes}
\subsubsection{Redes neuronales profundas (GANs, U-Net)}
\subsubsection{Aprendizaje por refuerzo para selección de filtros}
\subsection{Sistemas Multiagente}
\subsubsection{Definición y aplicaciones}
\subsubsection{Mecanismos de cooperación entre agentes}
\subsubsection{Modelos de comunicación en agentes (subastas, negociaciones)}

\section{Diseño del Sistema}
\subsection{Arquitectura General}
\subsection{Descripción de los Agentes}
\subsubsection{Agente de Eliminación de Ruido}
\subsubsection{Agente de Restauración de Bordes}
\subsubsection{Agente de Corrección de Color}
\subsection{Comunicación y Coordinación entre Agentes}
\subsection{Integración con Redes Neuronales}

\section{Implementación}
\subsection{Tecnologías Utilizadas}
\subsubsection{Python, NumPy, OpenCV, PIL}
\subsubsection{Redes neuronales con PyTorch/TensorFlow}
\subsubsection{Frameworks de Agentes (Mesa, JADE)}
\subsubsection{Redes distribuidas con sockets/MQTT}
\subsection{Proceso de Desarrollo}
\subsection{Pruebas y Evaluación}

\section{Resultados y Discusión}
\subsection{Restauración de Fotografías Antiguas}
\subsection{Evaluación del Desempeño del Sistema}
\subsection{Comparación con Métodos Tradicionales}

\section{Conclusiones y Trabajo Futuro}
\subsection{Conclusiones Generales}
\subsection{Limitaciones del Sistema}
\subsection{Líneas de Investigación Futura}

\end{document}
